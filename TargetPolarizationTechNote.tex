\title{Spent Nuclear Fuel Contributions to P15A Data Set}
\author{
        Donald Jones \\
        Temple University\\
 }
\date{\today}

\documentclass[12pt]{article}
\usepackage[english]{babel}
\usepackage{geometry} 
\geometry{letterpaper}
\usepackage{caption}
\usepackage{fullpage}
\usepackage{placeins}
\usepackage{graphicx}
\usepackage{amssymb}
\usepackage{amsmath}
\usepackage{color}
\begin{document}
\maketitle

\begin{abstract}
The future parity violation program in Hall A including the MOLLER experiment and the SOLID experimental program requires knowledge of the electron beam polarization at <0.5\% absolute uncertainty. One of the existing polarimeters in Hall A is the M\o ller polarimeter which utilizes electron electron scattering where the electron beam is scattered from the atomic electrons in a polarized iron foil target. A left-right spin-dependent scattering asymmetry exists from which the beam polarization can me determined using $A_{LR}=P_{target}P_{beam}\left<A_{zz}\right>$, where $A_{zz}$ is the analyzing power of the target (7/9 for exact 90 degree centet of mass scattering). In this document I concentrate on how well the target polarization can be known. Precision measurements of the magnetization of iron, nickel and cobalt were made over several decades from the 1930's to the 1970's. I try to determine how accurate these values are and what corrections we need for the conditions in Hall A. 
\end{abstract}

\section{Introduction}

%\begin{figure}
%\centering
%\captionsetup{width=5in}
%\includegraphics[width=5in]{}
%\caption{ Fractional }
%\label{fig:Xubo_spectrum}
%\end{figure}

\section{Methodology}\label{method}



\section{Conclusions}\label{conclusions}

\FloatBarrier
\bibliographystyle{abbrv}
\bibliography{}

\end{document}
