\title{Target Foil Polarization for M\o ller Polarimetry in Hall A}
\author{
        Donald Jones \\
        Temple University\\
 }
\date{\today}

\documentclass[12pt]{article}
\usepackage[english]{babel}
\usepackage{geometry} 
\geometry{letterpaper}
\usepackage{caption}
\usepackage{fullpage}
\usepackage{placeins}
\usepackage{graphicx}
\usepackage{amssymb}
\usepackage{amsmath}
\usepackage{color}
\begin{document}
\maketitle

\begin{abstract}
The future parity violation (PV) program in Hall A including the MOLLER experiment and the SOLID experimental program requires knowledge of the electron beam polarization at $<0.5\%$ absolute uncertainty. One of the existing polarimeters in Hall A is the M\o ller polarimeter which utilizes electron-electron scattering where the electron beam is scattered from the atomic electrons in a polarized iron foil target. A left-right spin-dependent scattering asymmetry exists from which the beam polarization can be determined using $A_{LR}=P^{target}P^{beam}\left<A_{zz}\right>$, where $A_{zz}$ is the analyzing power of the target (7/9 for exact 90 degree center of mass scattering). In this document I concentrate on how well the target polarization can be known. Precision measurements of the magnetization of iron, nickel and cobalt were made over several decades from the 1930's to the 1970's. I try to determine how accurate these values are and what corrections we need for the conditions in Hall A. 
\end{abstract}

\section{Introduction}
M\o ller (electron-electron) scattering at tree level in the center of mass (CM) system is given by
\begin{equation}
\frac{d\sigma}{d\Omega_{cm}}=\frac{\alpha^2}{s}\frac{\left(3+\cos^2\theta\right)^2}{\sin^4\theta}\left[1-P^{target}_{\ell}P^{beam}_{\ell}A_{\ell}(\theta)-P^{target}_tP^{beam}_tA_t(\theta)\cos\left(2\phi-\phi_{beam}-\phi_{target}\right)\right]
\label{eq:moller_cx}
\end{equation} 
where the subscripts $T$ and $L$ refer to transverse and longitudinal polarization respectively. In the center of mass at high energy, the Mendalstam variable $s$ is equal to $(2E_{CM})^2$. The CM scattering angle is $\theta$ and the azimuthal angle of the target (beam) polarization with respect to the electron beam is $\phi_{target(beam)}$. The analyzing powers for longitudinal and transverse polarization are given by
\begin{equation}
A_{\ell}(\theta)=\frac{\left(7+\cos^2\theta\right)\sin^2\theta}{\left(3+\cos^2\theta\right)^2}~~~\textrm{and}~~~A_t(\theta)=\frac{\sin^4\theta}{\left(3+\cos^2\theta\right)^2}.
\label{eq:analyzing_pow}
\end{equation}
$A_{\ell}$ is much larger than $A_t$ making M\o ller polarimetery much more sensitivity to longitudinal polarization. Since $A_{\ell}$ is a maximum for 90 degree CM scattering where $A_{\ell}=7/9$, the optics of the M\o ller polarimeter in Hall A are tuned to accept events near this maximum. For the setup in Hall A, the alignment is such that the transverse polarization of the target is essentially zero and I will not consider this term. Integrating the cross section over the acceptance of the detector gives 
\[
\sigma \propto\left(1-P^{target}_{\ell}P^{beam}_{\ell}\left<A_{zz}\right>\right),
\]
where $A_{zz}$ is the acceptance-weighted analyzing power $A_{\ell}$. We can now see that the left-right scattering asymmetry $A_{LR}$ is then given by 
\begin{equation}
A_{LR}=\frac{\sigma_R-\sigma_L}{\sigma_R+\sigma_L}=P^{target}_{\ell}P^{beam}_{\ell}\left<A_{zz}\right>,
\label{eq:A_LR}
\end{equation}
where $\sigma_{L(R)}$ are the cross sections for left (right) helicity electrons.

If $A_{zz}$ and the target polarization $P_{\ell}^{target}$ are known the beam polarization can be determined from the measured scattering asymmetry. If the beam polarization is to be known to better than 0.5\%, the target polarization must be accurately determined. The remainder of this document deals with issues for determining the target polarization.
\section{Foil Target Polarization}
The M\o ller polarimeter target consists of a set of thin foils magnetized out of plane parallel (or antiparallel) to the beam trajectory. The three ferromagnetic elements, Fe, Co and Ni are the obvious choices due to their relatively high magnetization and the precision with which the magnetization and the spin component of this is known. The default foil of choice has thus far been pure iron since its magnetization is known with the least relative error and because it has a relatively high Curie temperature, making it less sensitive to beam heating effects.
\begin{table}[h]
\begin{center}
\begin{tabular}{|r|l|l|l|}\hline
~&Fe&Co&Ni\\\hline
Z&26&27&28\\
Atomic Mass ($\mu$)&55.845(2)&58.933194(4)&58.6934(4)\\
Electron Configuration&[Ar]$4s^23d^6$&[Ar]$4s^23d^7$&[Ar]$4s^23d^8$ or $4s^13d^9$\\
Unpaired Electrons&2.2&1.72&0.6\\
Density near r.t. (g/cm$^3$)&7.874&8.900&8.902\\
$M_o$ at 0 K (emu/g)&221.7(2)&164.1(3)&58.6(2)\\
$g^{\prime}$&1.919(2)&1.850(4)&1.835(2)\\
Curie Temperature (K)& 1043&1400&358\\\hline
\end{tabular}
\end{center}
\caption{Properties of the three ferromagnetic elements.}
\end{table}

Although the magnetization of Fe and Ni are both known to high accuracy ($\sim0.2$ emu/g, since the magnetization of Fe is 3 ot 4 times larger, the relative error is much smaller. The low Curie temperature of Ni makes it susceptible to large (percent level) corrections from target heating effects. There are fewer published measurements of high precision on Co than on the other two ferromagnetic elements.

M\o ller polarimetery requires finding the average target electron polarization; however, magnetization measures the magnetic moment of the whole atom including the orbital and spin magnetic moments. Since we only want the spin component we need to find the faction of the magnetization that comes from spin. This is typically determined from precise measurements of the gyromagnetic ratio of an elemental sample. Thus, the final error on the target polarization will include uncertainties on both the determination of magnetization and of the spin component.

In the following sections I look at each of the three elements and determine what the total uncertainty would be if we used each of the three ferromagnetic elements as our target.

The issues facing us are follows:
\begin{itemize}
\item{Through the years from 1930-1980 many precise measurements have been made of the magnetization and gyromechanical properties of these elements; however, they do not necessarily agree within error. Sometimes the errors quoted are not realistic given the systematic disagreement in the data. The sources of systematic difference are often not known and yet results are averaged together and the final error quoted as the statistical variation.}
\item{No one ever mentions nuclear contributions from proton and neutrons. The nuclear magneton is smaller than the Bohr magneton by a factor of $m_e/m_p\sim0.05\%$. Fortunately, the main isotopes that make up iron and nickel are even-even and have spinless nuclei, but for Co the average is 4.6 nuclear magnetons taking us above the 0.1\% level we care about.}
\item{How well do we know the corrections needed to take us from the field and temperature values of the values in the literature to the conditions in our polarimeter?}
\item{Through the past century measurement of constants have become more precise and have changed. Examples of constants used in determining quoted magnetization and gyromagnetic data in the literature are the density of elements, the charge to mass ratio of the electron, and the Bohr magneton. Different groups use different values. Where it is possible the quoted values should be scaled to reflect currently accepted values for constants or an appropriate uncertainty assigned.}
\item{Experiments measuring properties of these ferromagnetic elements used different levels of purity. What level of uncertainty should be assigned to account for the effects of impurities?}
\end{itemize}

\
%\centering
%\captionsetup{width=5in}
%\includegraphics[width=5in]{}
%\caption{ Fractional }
%\label{fig:Xubo_spectrum}
%\end{figure}

\section{Methodology}\label{method}
Target polarization is determined from measurements of the saturation magnetization of pure iron. Another term used in the literature is ``spontaneous magnetization,'' which as the name implies refers to the magnetic moment of a material that spontaneously arises with no applied field. In ferromagnetic materials the magnetic momemts of the electrons tend to spontaneously align in a given direction. However, due to energy considerations, domains which are small regions of aligned spin, tend to form in such a way so that the total spin averaged across many domains at the macroscopic level is far below the saturation level and may be 0. In the presence of an applied magnetic field, the domain boundaries shift with more electron magentic moments aligning along the direction of the field. As the applied field is increased, eventually the material will reach magnetic saturation where all the spins are aligned along the direction of the applied field. Thus, the saturation magnetization and the spontaneous magnetization are numerically equal although the spontaneous magnetization cannot be measured at room temperature due to domain formation. 

Spontaneous magnetization is a function of temperature and applied field and for this reason it is often given as $M_{0}$, the value of saturation magnetization extrapolated to zero applied field at $T = 0^{\circ}$ K. However, experiments measure the magnetization at temperatures above 0 K with applied fields. For temperatures well below the Curie temperature the magnetization has been shown to roughly follow the $T^{3/2}$ law of Bloch given as \cite{Bloch1930}
\begin{equation}
M_s(T) = M_0(1-a_{3/2}T^{3/2}),
\label{eq:bloch}
\end{equation}
where $M_0$ is the saturation magnetization at 0 K and $a_{3/2}$ is an emperically determined constant. At higher fields additional terms to account for spin waves and the action of conduction electrons are required\cite{PauthenetNov1982}. Pauthenet expresses the magnetization as a function of temperature and applied field as follows:\cite{PauthenetMar1982,PauthenetNov1982}
\begin{equation}
M_s(H_i, T) = M_s(T)+A(T)H_i^{1/2}+B(T)H_i,
\label{eq:pauthenet}
\end{equation}
where $M_s$ is given be Eq \ref{eq:bloch} and A(T) and B(T) are functions of temperature and can be extracted from fits to data of magnetization versus internal field at a constant temperature. Pauthenet utilizes fits to his data to give a numerical expression for magnetization as a function of internal magnetic field and temperature (see equation 9 and Table 1 from \cite{PauthenetMar1982}). Corrections for differences in temperature and internal field made will come from equation 9 in \cite{PauthenetMar1982}.

It is important to note the difference between internal field and applied field. The magnetization of an object at a particular temperature and applied field is not just a function of its elemental composition\footnote{I am ignoring magnetic history which can also potentially affect the magnetization one observes for a given applied field.}. Other factors that affect the magnetization are
DEAL HERE WITH $H=H_i+4\pi M$
\begin{itemize}
\item{Shape anisotropy:needles different from spheres}
\item{Crystal anisotropy: depends on whether polycrysalline or monocystalline and alignment with axes.}
\item{History: some crystals have more than one possible crystal structure with different magnetizations. Their history of heating/cooling and annealing can have an effect on their magnetic properties. Cobalt, for example, has a face-centered cubic crystal structure above 690 K which is unstable below that temperature. However, the exact crystal structure below 690 K (and by extension the magnetization) depends upon the grain size and the annealing process used to prepare it \cite{Owen1954}.}
\end{itemize}

 can be conveniently express in terms of the saturation magnetizat the For the M\o ller polarimeter in particular, the temperature of the iron foil is at $\sim300^{\circ}$~K and in an applied magnetic field $\sim$4~T. The higher the temperature, the less the spins tend to align,

For pure iron, at saturation, the magnetic moment per unit volume is $\sim$2.2 T. The saturation magnetization is the same as the spontaneous magnetization
 
These measurements are performed at a variety of applied magnetic fields and temperature and typically expressed in terms of the saturation magnetization $M_o$ which is the extrapolation to 0 zero applied field at absolute 0 temperature\cite{Crangle1971}. Although different methods are used to measure the saturation magnetization, they broadly break down into two categories. 
\begin{itemize}
\item{Force method: small ellipsoid sample of the element of interest is placed in a precisely determined field gradient. With a proper setup, the force on the sample by the magnetic field can be shown to be the product of the magnetization and the magnetic field gradient. Thus the magnetization is given as the force divided by the field gradient.}
\item{Induction method: a sample is placed into a magnetic field and its presence creates a magnetic moment that is measured in pickup coils.}
\end{itemize}      
Although the experimental methods can be thus broadly categorized, each individual experiment takes a slightly different approach to measurement and calibration.

\section{Conclusions}\label{conclusions}

\FloatBarrier
%\bibliographystyle{abbrv}
\bibliographystyle{unsrt}
\bibliography{bibliography}

\end{document}
