%% 
%% Copyright 2007-2019 Elsevier Ltd
%% 
%% This file is part of the 'Elsarticle Bundle'.
%% ---------------------------------------------
%% 
%% It may be distributed under the conditions of the LaTeX Project Public
%% License, either version 1.2 of this license or (at your option) any
%% later version.  The latest version of this license is in
%%    http://www.latex-project.org/lppl.txt
%% and version 1.2 or later is part of all distributions of LaTeX
%% version 1999/12/01 or later.
%% 
%% The list of all files belonging to the 'Elsarticle Bundle' is
%% given in the file `manifest.txt'.
%% 

%% Template article for Elsevier's document class `elsarticle'
%% with numbered style bibliographic references
%% SP 2008/03/01
%%
%% 
%%
%% $Id: elsarticle-template-num.tex 168 2019-02-25 07:15:41Z apu.v $
%%
%%
\documentclass[preprint,12pt]{elsarticle}
\usepackage{xcolor}
\pagecolor{white}
\usepackage{hyperref}
\usepackage{subcaption}
%% Use the option review to obtain double line spacing
%% \documentclass[authoryear,preprint,review,12pt]{elsarticle}

%% Use the options 1p,twocolumn; 3p; 3p,twocolumn; 5p; or 5p,twocolumn
%% for a journal layout:
%% \documentclass[final,1p,times]{elsarticle}
%% \documentclass[final,1p,times,twocolumn]{elsarticle}
%% \documentclass[final,3p,times]{elsarticle}
%% \documentclass[final,3p,times,twocolumn]{elsarticle}
%% \documentclass[final,5p,times]{elsarticle}
%% \documentclass[final,5p,times,twocolumn]{elsarticle}

%% For including figures, graphicx.sty has been loaded in
%% elsarticle.cls. If you prefer to use the old commands
%% please give \usepackage{epsfig}
\usepackage{wrapfig}
\usepackage[section]{placeins}
\graphicspath{{figures/}}
%% The amssymb package provides various useful mathematical symbols
\usepackage{amssymb}
%% The amsthm package provides extended theorem environments
\usepackage{amsthm}
\usepackage{amsmath}

%% The lineno packages adds line numbers. Start line numbering with
%% \begin{linenumbers}, end it with \end{linenumbers}. Or switch it on
%% for the whole article with \linenumbers.
\usepackage{lineno}

\journal{Nuclear Physics A}

\begin{document}
%\linenumbers


%% Title, authors and addresses

%% use the tnoteref command within \title for footnotes;
%% use the tnotetext command for theassociated footnote;
%% use the fnref command within \author or \address for footnotes;
%% use the fntext command for theassociated footnote;
%% use the corref command within \author for corresponding author footnotes;
%% use the cortext command for theassociated footnote;
%% use the ead command for the email address,
%% and the form \ead[url] for the home page:
%% \title{Title\tnoteref{label1}}
%% \tnotetext[label1]{}
%% \author{Name\corref{cor1}\fnref{label2}}
%% \ead{email address}
%% \ead[url]{home page}
%% \fntext[label2]{}
%% \cortext[cor1]{}
%% \address{Address\fnref{label3}}
%% \fntext[label3]{}

\title{Target Heating Correction }

%% use optional labels to link authors explicitly to addresses:
%% \author[label1,label2]{}
%% \address[label1]{}
%% \address[label2]{}

%% main text
\section{Introduction }
\begin{align*}
\nabla^2T(r)&=-\frac{\alpha}{\sqrt{2\pi r_b^2}} e^{-r^2/2r_b^2}\\
\frac{1}{r}\frac{\partial }{\partial r}\left( r\frac{\partial T(r)}{\partial r}\right)&=-\frac{\alpha}{\sqrt{2\pi r_b^2}} e^{-r^2/2r_b^2}\\
\frac{\partial }{\partial r}\left( r\frac{\partial T(r)}{\partial r}\right)&=-\frac{\alpha r}{\sqrt{2\pi r_b^2}}  e^{-r^2/2r_b^2}\\
\end{align*}
Integrate both sides and set $a=\alpha/\sqrt{2\pi}$.
\begin{align*}
 r\frac{\partial T(r)}{\partial r}&=ar_be^{-r^2/2r_b^2}+C\\
 \frac{\partial T(r)}{\partial r}&=\frac{a r_b}{r}e^{-r^2/2r_b^2}+\frac{C}{r}\\
\end{align*}
Integrate again over a region of $0<r<r_0$:
\begin{align*}
 T(r^{\prime})\vert_r^{r_0}=\Delta T&=\int _{r}^{r_0} dr^{\prime}\left(\frac{ ar_b}{r^{\prime}}e^{-r^{\prime 2}/2r_b^2}+\frac{C}{r^{\prime}}\right)\\
 ~&=\frac{ar_b}{2} \left[\textrm{Ei}(-r_0^2/2r_b^2)-\textrm{Ei}(-r^2/2r_b^2)\right]+C\textrm{ln}(r_0/r)
 \end{align*}

Or for a uniformly rastered heat load over a circular area radius R
\begin{align*}
 \Delta T&=\int _{r}^{r_0} dr\left(\frac{\alpha\Theta(R-r)}{\pi R^2r}+\frac{C}{r}\right)\\
 \end{align*}
for $r>R$ this is just $T(r)=C$ln$(r/r_0)$. For $r<R$
\begin{align*}
 T(r)&=\int _{r}^{r_0} dr\left(\frac{\alpha\Theta(R-r)}{\pi R^2r}+\frac{C}{r}\right)\\
 ~&=\frac{\alpha}{\pi R^2}\ln{(R/r)}+C\ln{(r_0/r)}
 \end{align*}
Now impose B.C.s $T(r_0)=T_0$.
\begin{align*}
b
\end{align*}
\FloatBarrier
\begin{tabular}{| l c c |}\hline
Source&Value&$\delta$P/P(\%)\\ \hline
Foil Polarization &0.08005&0.57\\
High Current Extrapolation&$-$&0.50\\
Null Asymmetry (Cu Foil)&0.00\%&0.22\\
Beam Bleedthrough&$-$&0.18\\
$A_{zz}$&0.75421&0.16\\
Dead Time Correction&0.148\%&0.15\\
Other&$-$&0.13\\\hline
~&\textbf{Total}&\textbf{0.85}\\\hline
\end{tabular}
%% The Appendices part is started with the command \appendix;
%% appendix sections are then done as normal sections
%% \appendix

%% \section{}
%% \label{}

%% If you have bibdatabase file and want bibtex to generate the
%% bibitems, please use
\bibliographystyle{elsarticle-num} 
%%  \bibliography{<your bibdatabase>}

%% else use the following coding to input the bibitems directly in the
%% TeX file.

\bibliography{bibliography}
\end{document}
%\endinput
%%
%% End of file `elsarticle-template-num.tex'.
� 2020 GitHub, Inc.
Terms
Privacy
Security
Status
Help
Contact GitHub
Pricing
API
Training
Blog
About
